% BASIC SETTINGS
\documentclass[a4paper,12pt]{article} % Set paper size and document type
\usepackage{lmodern} % Use a slightly nicer looking font

% Change margins - default margins are too broad
\usepackage[margin=20mm]{geometry}

% SOURCE CODE LISTING SETTINGS 
% https://en.wikibooks.org/wiki/LaTeX/Source_Code_Listings
\usepackage{listings}
\usepackage{color}

\definecolor{mygreen}{rgb}{0,0.6,0}
\definecolor{mygray}{rgb}{0.5,0.5,0.5}
\definecolor{mymauve}{rgb}{0.58,0,0.82}

\lstset{ %
  backgroundcolor=\color{white},   % choose the background color; you must add \usepackage{color} or \usepackage{xcolor}
  basicstyle=\footnotesize,        % the size of the fonts that are used for the code
  breakatwhitespace=false,         % sets if automatic breaks should only happen at whitespace
  breaklines=true,                 % sets automatic line breaking
  captionpos=b,                    % sets the caption-position to bottom
  commentstyle=\color{mygreen},    % comment style
  deletekeywords={...},            % if you want to delete keywords from the given language
  escapeinside={\%*}{*)},          % if you want to add LaTeX within your code
  extendedchars=true,              % lets you use non-ASCII characters; for 8-bits encodings only, does not work with UTF-8
  frame=single,	                   % adds a frame around the code
  keepspaces=true,                 % keeps spaces in text, useful for keeping indentation of code (possibly needs columns=flexible)
  keywordstyle=\color{blue},       % keyword style
  otherkeywords={*,...},           % if you want to add more keywords to the set
  numbers=left,                    % where to put the line-numbers; possible values are (none, left, right)
  numbersep=5pt,                   % how far the line-numbers are from the code
  numberstyle=\tiny\color{mygray}, % the style that is used for the line-numbers
  rulecolor=\color{black},         % if not set, the frame-color may be changed on line-breaks within not-black text (e.g. comments (green here))
  showspaces=false,                % show spaces everywhere adding particular underscores; it overrides 'showstringspaces'
  showstringspaces=false,          % underline spaces within strings only
  showtabs=false,                  % show tabs within strings adding particular underscores
  stepnumber=2,                    % the step between two line-numbers. If it's 1, each line will be numbered
  stringstyle=\color{mymauve},     % string literal style
  tabsize=2,	                   % sets default tabsize to 2 spaces
  title=\lstname                   % show the filename of files included with \lstinputlisting; also try caption instead of title
}

% PREPARE TITLE
\title{\textbf{Homework \#2 - Loops}}
\author{Name: }
\date{} % Hide the date

% START DOCUMENT
\begin{document}

\maketitle % Insert the title

\section{For loops}

Just like Python, C++ has both "while" and "for" loops. A "for" loop in C++ is written like this:

\vspace{5mm}
\begin{lstlisting}[language=C++]
#include <iostream>

using namespace std;

int main() {

	for (int i = 0; i < 10; i++) {
		cout << i << endl;
	}

	return 0;
}
\end{lstlisting}

\noindent
This code will print out the numbers from 0 - 9. In Python, you'd print the numbers from 0 - 9 like this:

\vspace{5mm}
\begin{lstlisting}[language=Python]
for i in range(0,10):
	print i
\end{lstlisting}

\noindent
What if we wanted to go through a list of both numbers and words? In Python, we could easily do that like this:

\vspace{5mm}
\begin{lstlisting}[language=Python]
for i in ['banana', 'apple', 36]
	print i
\end{lstlisting}

\noindent
Unfortunately, there is nothing like this in C++. For loops can only count up or down, they can't actually go "item by item" through a list containing different types of data. This is one of the big differences between C++ and Python.

\clearpage

\section{While loops}

While loops in C++ are a lot closer to their Python counterparts. A while loop looks like this:

\vspace{5mm}
\begin{lstlisting}[language=C++]
#include <iostream>

using namespace std;

int main() {

	int i = 0;
	while (i < 10) {
		cout << i << endl;
		i++;
	}
	
	return 0;
}
\end{lstlisting}

\noindent
One new thing here is the "i++". In C++ this means "add 1 to i". It's a bit like doing "i += 1" in Python...it's just shorter than writing "i = i + 1" and programmers love keeping things short!\\

\section{Now You Try}

\subsection{Summing up numbers}

Write a short C++ program which asks for a number, then adds up all the numbers from 1 up to that number. The output should be something like this:

\begin{verbatim}
Enter a number: 5
The sum is: 15
\end{verbatim}

\subsection{Countdown}

Try writing a short program which counts from -1 to -99 by twos, like this:

\begin{verbatim}
-1
-3
-5
-7
...
-99
\end{verbatim}

\noindent
When should the loop stop? How much should you add or subtract from "i" each time the loop runs? 


\end{document}