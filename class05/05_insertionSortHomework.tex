% BASIC SETTINGS
\documentclass[a4paper,12pt]{article} % Set paper size and document type
\usepackage{lmodern} % Use a slightly nicer looking font

% Change margins - default margins are too broad
\usepackage[margin=20mm]{geometry}

% SOURCE CODE LISTING SETTINGS 
% https://en.wikibooks.org/wiki/LaTeX/Source_Code_Listings
\usepackage{listings}
\usepackage{color}

\definecolor{mygreen}{rgb}{0,0.6,0}
\definecolor{mygray}{rgb}{0.5,0.5,0.5}
\definecolor{mymauve}{rgb}{0.58,0,0.82}

\lstset{ %
  backgroundcolor=\color{white},   % choose the background color; you must add \usepackage{color} or \usepackage{xcolor}
  basicstyle=\footnotesize,        % the size of the fonts that are used for the code
  breakatwhitespace=false,         % sets if automatic breaks should only happen at whitespace
  breaklines=true,                 % sets automatic line breaking
  captionpos=b,                    % sets the caption-position to bottom
  commentstyle=\color{mygreen},    % comment style
  deletekeywords={...},            % if you want to delete keywords from the given language
  escapeinside={\%*}{*)},          % if you want to add LaTeX within your code
  extendedchars=true,              % lets you use non-ASCII characters; for 8-bits encodings only, does not work with UTF-8
  frame=single,	                   % adds a frame around the code
  keepspaces=true,                 % keeps spaces in text, useful for keeping indentation of code (possibly needs columns=flexible)
  keywordstyle=\color{blue},       % keyword style
  otherkeywords={*,...},           % if you want to add more keywords to the set
  numbers=left,                    % where to put the line-numbers; possible values are (none, left, right)
  numbersep=5pt,                   % how far the line-numbers are from the code
  numberstyle=\tiny\color{mygray}, % the style that is used for the line-numbers
  rulecolor=\color{black},         % if not set, the frame-color may be changed on line-breaks within not-black text (e.g. comments (green here))
  showspaces=false,                % show spaces everywhere adding particular underscores; it overrides 'showstringspaces'
  showstringspaces=false,          % underline spaces within strings only
  showtabs=false,                  % show tabs within strings adding particular underscores
  stepnumber=2,                    % the step between two line-numbers. If it's 1, each line will be numbered
  stringstyle=\color{mymauve},     % string literal style
  tabsize=2,	                   % sets default tabsize to 2 spaces
  title=\lstname                   % show the filename of files included with \lstinputlisting; also try caption instead of title
}

% PREPARE TITLE
\title{\textbf{Homework \#5 - Sorting}}
\author{Name: }
\date{} % Hide the date

% START DOCUMENT
\begin{document}

\maketitle % Insert the title

\section{Intro}

You've now learned enough C++ that we can start talking about the idea of an \textbf{algorithm}. An algorithms is the way you do something. For instance, an algorithm for washing dishes might looks like:

\begin{enumerate}
\item Turn on sink
\item Put soap on sponge
\item Pick up dish (bowl, plate, fork, spoon, knife, or cup)
\item Wash with sponge
\item Dry
\item Go back to step 3
\item Stop when there are no more dishes
\end{enumerate}

\noindent
Actually, an algorithm is not too different from a program. It's a list of instructions that tell you \textbf{how} to do something. The difference is that usually an algorithm only talks about how to do one very specific job, usually something mathematical like searching or sorting. Programs often have a lot of \textbf{other} work to do like opening and closing files, waiting for users to click on things, etc... and algorithms do not deal with that. 

\section{Sorting}

We talked a little bit about sorting today in class. There are many different ways to sort, and \textbf{how} you sort can affect how quickly you can sort. For homework, we'll try to write two simple sorting algorithms, Bubble Sort and Insertion Sort.\\

\noindent
For most problems, Bubble Sort is very slow, but there are some times when it is the right choice. For instance, if most of your lists look like this:

\begin{verbatim}
1 2 3 4 5 6 8 7 9 10 11 12
\end{verbatim}

\noindent
Notice that only 8 and 7 are not in the right order in this list. For lists that are \textbf{almost} in the right order already with some things only one or two places away from the correct position, then Bubble Sort is a good choice. We will talk more about this later in the class. Not only does the \textbf{algorithm} you use determine how fast you can solve a problem, but sometimes the problem can help you decide which algorithm is the right one to use. Not every kind of data is the same, so there is no \textbf{best} algorithm for all situations.\\ 

\noindent
Remember, in algorithms there is \textbf{no substitute for understanding the problem you are trying to solve}. 

\subsection{Bubble Sort}

Bubble Sort is perhaps the simplest sorting algorithm. Bubble sort for a list of numbers works like this:

\begin{enumerate}
\item Look at the 1st and 2nd numbers in the list
\item If the numbers are not in the right order, flip them
\item Look at the 2nd and 3rd numbers in the list
\item If the numbers are not in the right order, flip them
\item Continue to the end of the list, flipping any out of order number pairs
\item Start again from the beginning of the list
\item If you go through the whole list without flipping any numbers, stop
\end{enumerate}

\noindent
You can use the code in emptySort.cpp to get started. You just need to write the sorting code in the middle. The code to create and print the array(s) is already written for you in that file. 

\subsection{Insertion Sort}

Below is the pseudo-code for insertion sort. Your job is to turn this into a real C++ program which you can show me in the next class. 

\vspace{5mm}
\lstinputlisting[language=C++]{insertionSortPseudo.cpp}

\noindent
Think about how you could convert this into a real C++ program. How does it work? What happens when this code finds two numbers that are already in order? What about two numbers that are \textbf{not} in order? You can start with the C++ code on the next page (emptySort.cpp)

\clearpage

\vspace{5mm}
\lstinputlisting[language=C++, firstline=6]{emptySort.cpp}
\vspace{5mm}

\noindent
Your new code should go in the middle, between the two for loops. 

\subsection{Ok...what about big to small?}

Write a C++ program that sorts in reverse. Big numbers first, small numbers last. You can start with the code you wrote for the last question.

\subsection{A final challenge}

How would you sort words from A-Z? Think about an algorithm to do this. You don't need to write real code, but please write the \textbf{pseudocode} for your algorithm here (or on the back of the page if you run out of room): 

\begin{enumerate}
\item
\item
\item
\item
\item
\item
\item
\end{enumerate}

\end{document}