% BASIC SETTINGS
\documentclass[a4paper,12pt]{article} % Set paper size and document type
\usepackage{lmodern} % Use a slightly nicer looking font

% Change margins - default margins are too broad
\usepackage[margin=20mm]{geometry}

% SOURCE CODE LISTING SETTINGS 
% https://en.wikibooks.org/wiki/LaTeX/Source_Code_Listings
\usepackage{listings}
\usepackage{color}

\definecolor{mygreen}{rgb}{0,0.6,0}
\definecolor{mygray}{rgb}{0.5,0.5,0.5}
\definecolor{mymauve}{rgb}{0.58,0,0.82}

\lstset{ %
  backgroundcolor=\color{white},   % choose the background color; you must add \usepackage{color} or \usepackage{xcolor}
  basicstyle=\footnotesize,        % the size of the fonts that are used for the code
  breakatwhitespace=false,         % sets if automatic breaks should only happen at whitespace
  breaklines=true,                 % sets automatic line breaking
  captionpos=b,                    % sets the caption-position to bottom
  commentstyle=\color{mygreen},    % comment style
  deletekeywords={...},            % if you want to delete keywords from the given language
  escapeinside={\%*}{*)},          % if you want to add LaTeX within your code
  extendedchars=true,              % lets you use non-ASCII characters; for 8-bits encodings only, does not work with UTF-8
  frame=single,	                   % adds a frame around the code
  keepspaces=true,                 % keeps spaces in text, useful for keeping indentation of code (possibly needs columns=flexible)
  keywordstyle=\color{blue},       % keyword style
  otherkeywords={*,...},           % if you want to add more keywords to the set
  numbers=left,                    % where to put the line-numbers; possible values are (none, left, right)
  numbersep=5pt,                   % how far the line-numbers are from the code
  numberstyle=\tiny\color{mygray}, % the style that is used for the line-numbers
  rulecolor=\color{black},         % if not set, the frame-color may be changed on line-breaks within not-black text (e.g. comments (green here))
  showspaces=false,                % show spaces everywhere adding particular underscores; it overrides 'showstringspaces'
  showstringspaces=false,          % underline spaces within strings only
  showtabs=false,                  % show tabs within strings adding particular underscores
  stepnumber=2,                    % the step between two line-numbers. If it's 1, each line will be numbered
  stringstyle=\color{mymauve},     % string literal style
  tabsize=2,	                   % sets default tabsize to 2 spaces
  title=\lstname                   % show the filename of files included with \lstinputlisting; also try caption instead of title
}

% PREPARE TITLE
\title{\textbf{Homework \#7 - Structs and Classes}}
\author{Name: }
\date{} % Hide the date

% START DOCUMENT
\begin{document}

\maketitle % Insert the title

\section{Structs}

One of the \textbf{big ideas} in C++ is something called OOP or \textbf{object oriented programming}. There are lots of real world programming problems that are best solved by putting a lot of data together into a new \textbf{type} that you make yourself. Grouping related data together can make your programs easier to design and easier to read and modify. In C++, you can group data together with a \textbf{struct}. A struct looks like this:

\vspace{5mm}
\lstinputlisting[language=C++,firstline=11,lastline=15]{structs.cpp}

\noindent
This struct would make it very easy to keep track of a list of students in an array, because now a single array can contain all this data together in one place (no need for a "student names" array and a "student ages" array, etc...). Creating an array to hold 3 students is now easy:

\vspace{5mm}
\lstinputlisting[language=C++,firstline=19,lastline=20]{structs.cpp}

\noindent
Let's say you wanted to change the name of the first student in the array, students[0]. You'd just write:

\vspace{5mm}
\begin{lstlisting}[language=C++]
students[0].name = "Frank";
\end{lstlisting}

\noindent
You could of course write a loop to fill in all the elements of the array, as well:

\vspace{5mm}
\lstinputlisting[language=C++,firstline=23,lastline=32]{structs.cpp}

\section{Classes}

Letting us group data into new types like this is a great feature, but C++ goes farther than that. It also allows us to put \textbf{functions} together with our data. Now, our data \textbf{and} the functions we need to work with it can all be in one place. We use classes to do this. Here's an example of the Student struct rewritten as a class:

\vspace{5mm}
\lstinputlisting[language=C++,firstline=11,lastline=31]{classes.cpp}

\noindent
Notice that the work of printing out everything inside the class can now be hidden inside the class. If I want to print out all the details for students[1] I would just say:

\vspace{5mm}
\begin{lstlisting}[language=C++]
students[1].printData();
\end{lstlisting}

\noindent
To do this with a struct, I'd have to write:

\vspace{5mm}
\begin{lstlisting}[language=C++]
cout << "Name: " << students[1].name << endl;
cout << "Age: " << students[1].age << endl;
cout << "Score: " << students[1].score << endl;
\end{lstlisting}

\noindent
Having functions saved inside with the data makes classes much more powerful than structs. There are lots of situations - in fact - where classes are very helpful because they can hide the details of their insides behind "set()" and "get()" functions, so programmers don't need to know what's inside a class in order to use it. These "get()" and "set()" functions can also make sure that anybody using the class is changing the class's data in the correct way. 

\clearpage

\section{Now you try...}

\subsection{Using Structs}

Write a struct for use in a Zoo. You should make a struct that can store:

\begin{itemize}
\item An animal's name (ex: Billy)
\item What it eats (ex: Bamboo)
\item What type it is (ex: Panda)
\item It's age (ex: 12)
\item How many legs it has (ex: 4)
\end{itemize}

\noindent
Once you have written this struct, you should make an array big enough to hold 3 animals, and make a loop to let the user enter the information for these 3 animals (name, age, legs, etc...). You can look at \textbf{structs.cpp} for an example.

\subsection{Using Classes}

Change your Zoo struct into a Zoo class. You should add functions to \textbf{set} the animal's name/food/type/age/legs and another type to \textbf{get} these things and print them out. Start with the code in \textbf{classes.cpp}.

\subsection{Final Challenge}

Make a class called "Array" that can hold an array. The class should have functions to change what's stored in the array and print out the array. Something like this would be a good start:

\vspace{5mm}
\begin{lstlisting}[language=C++]
class Array {
	// Data
	private:
		vector<int> array; 
	    
	// Methods (functions inside the class)
	public:
		void printArray() {
			// YOUR CODE HERE
		}
		
		void changeItem(int index, int newValue) {
			// YOUR CODE HERE
		}
	
};
\end{lstlisting}

\end{document}