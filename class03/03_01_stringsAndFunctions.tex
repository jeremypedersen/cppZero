% BASIC SETTINGS
\documentclass[a4paper,12pt]{article} % Set paper size and document type
\usepackage{lmodern} % Use a slightly nicer looking font

% Change margins - default margins are too broad
\usepackage[margin=20mm]{geometry}

% SOURCE CODE LISTING SETTINGS 
% https://en.wikibooks.org/wiki/LaTeX/Source_Code_Listings
\usepackage{listings}
\usepackage{color}

\definecolor{mygreen}{rgb}{0,0.6,0}
\definecolor{mygray}{rgb}{0.5,0.5,0.5}
\definecolor{mymauve}{rgb}{0.58,0,0.82}

\lstset{ %
  backgroundcolor=\color{white},   % choose the background color; you must add \usepackage{color} or \usepackage{xcolor}
  basicstyle=\footnotesize,        % the size of the fonts that are used for the code
  breakatwhitespace=false,         % sets if automatic breaks should only happen at whitespace
  breaklines=true,                 % sets automatic line breaking
  captionpos=b,                    % sets the caption-position to bottom
  commentstyle=\color{mygreen},    % comment style
  deletekeywords={...},            % if you want to delete keywords from the given language
  escapeinside={\%*}{*)},          % if you want to add LaTeX within your code
  extendedchars=true,              % lets you use non-ASCII characters; for 8-bits encodings only, does not work with UTF-8
  frame=single,	                   % adds a frame around the code
  keepspaces=true,                 % keeps spaces in text, useful for keeping indentation of code (possibly needs columns=flexible)
  keywordstyle=\color{blue},       % keyword style
  otherkeywords={*,...},           % if you want to add more keywords to the set
  numbers=left,                    % where to put the line-numbers; possible values are (none, left, right)
  numbersep=5pt,                   % how far the line-numbers are from the code
  numberstyle=\tiny\color{mygray}, % the style that is used for the line-numbers
  rulecolor=\color{black},         % if not set, the frame-color may be changed on line-breaks within not-black text (e.g. comments (green here))
  showspaces=false,                % show spaces everywhere adding particular underscores; it overrides 'showstringspaces'
  showstringspaces=false,          % underline spaces within strings only
  showtabs=false,                  % show tabs within strings adding particular underscores
  stepnumber=2,                    % the step between two line-numbers. If it's 1, each line will be numbered
  stringstyle=\color{mymauve},     % string literal style
  tabsize=2,	                   % sets default tabsize to 2 spaces
  title=\lstname                   % show the filename of files included with \lstinputlisting; also try caption instead of title
}

% PREPARE TITLE
\title{\textbf{Homework \#3 - Handling Strings}}
\author{Name: }
\date{} % Hide the date

% START DOCUMENT
\begin{document}

\maketitle % Insert the title

\section{String I/O}

In Python, working with strings was easy! We never had to worry about how much space a string took up or what we would do if there were spaces in it. In C++ we do have to worry about these things. When we want to handle strings, we have to create a variable with the \textbf{string} type. Here's an example:

\vspace{5mm}
\lstinputlisting[language=C++, firstline=6]{stringIO_1.cpp}

\noindent
This will work fine for single-word inputs like "steve" or "jeremy" but would fail on "jeremy pedersen" or "i am a hippo" because "cin" treats words with spaces separating them as separate strings. To deal with this, C++ provides us a function called "getline()". It works a lot more like Python's raw\_input():

\vspace{5mm}
\lstinputlisting[language=C++, firstline=6]{stringIO_2.cpp}

\section{Functions}

Just like Python, C++ has a way for us to make and use functions. Functions in C++ look like this (pay attention to the top part of the program):

\vspace{5mm}
\lstinputlisting[language=C++, firstline=6]{function.cpp}

\noindent Instead of using "def" we start our functions by saying what \textbf{type} of data they will return. And instead of using tabs to say what belongs inside the function, we use the braces "\{" and "\}". Otherwise, using functions is very much like you are used to from Python.\\

\section{Now You Try}

\subsection*{Multi-word Strings}

Write a short C++ program which can ask for a name and print it back out. It should work for names that are \textbf{more than one word long} (i.e. names that have spaces in them such as "Kevin Gu" or "Jeremy Pedersen"). It should look like this when it runs:

\begin{verbatim}
Enter a name: Jeremy Pedersen
Hello, Jeremy Pedersen
\end{verbatim}

\subsection*{Inputting numbers}

Write a program which can change temperatures from C to F. For instance, if you type in 100, the program should give you the answer 212 (because 100 C = 212 F). You can use the formula $F = {9 \over 5} \cdot C + 32$ to do the conversion from C to F. You should put the code to change temperature \textbf{inside a function} like this one:

\vspace{5mm}
\begin{lstlisting}[language=C++]
float tempConv(float c) {
	f = // Your code here
	return f;
}
\end{lstlisting}

\end{document}